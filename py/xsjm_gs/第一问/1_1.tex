\documentclass{article}
\usepackage{amsmath}
\usepackage{amsfonts}
\usepackage{graphicx}
\usepackage{geometry}
\usepackage{ctex}
\geometry{a4paper,scale=0.8}

\title{使用泊松分布计算抽样检测问题}
\author{企业零配件次品率检测}
\date{}

\begin{document}

\maketitle

\section{问题背景}
某企业生产一种畅销电子产品,需采购两种零配件。在装配成品时,只要其中一个零配件不合格,则成品不合格。企业通过抽样检测零配件的次品率来决定是否接收供应商的零配件。在以下两种情况下,企业应如何确定抽样检测方案?

\begin{itemize}
    \item 95\% 的信度下认定零配件次品率超过标称值,则拒收该批零配件。
    \item 90\% 的信度下认定零配件次品率不超过标称值,则接收该批零配件。
\end{itemize}

\section{泊松分布模型}
假设供应商的次品率为 \( p \),企业从供应商处抽取 \( n \) 个零配件进行检测。泊松分布适用于样本量大且次品率较低的场景。泊松分布的参数 \( \lambda \) 为:

\[
\lambda = n \times p_0
\]

其中,\( p_0 \) 为供应商声明的次品率(标称值),\( n \) 为抽样数。

泊松分布的概率质量函数(PMF)为:

\[
P(X = k) = \frac{\lambda^k e^{-\lambda}}{k!}
\]

其中,\( X \) 为次品数量,\( k \) 为检测到的次品个数,\( \lambda \) 为预期的次品数量。

\section{计算步骤}

\subsection{确定拒收和接收的条件}

1. **95\% 信度下的拒收条件**:

    - 零假设:\( p \leq p_0 \)(零配件的次品率不超过 10\%)。
    - 备择假设:\( p > p_0 \)(零配件的次品率超过 10\%)。
    - 在 95\% 的信度下拒收该批零配件,即找到 \( k \) 值使得:
    
    \[
    P(X \geq k) = 1 - P(X < k) < \alpha_0
    \]
    
    其中,\( \alpha_0 = 0.05 \)。
    
2. **90\% 信度下的接收条件**:
    
    - 在 90\% 信度下接收,即找到 \( k \) 值使得:
    
    \[
    P(X \leq k) > 1 - \beta_0
    \]
    
    其中,\( \beta_0 = 0.10 \)。
    
\subsection{计算次品数量的范围}

根据泊松分布的概率质量函数,可以计算不同 \( k \) 值下的次品数量概率。定义 \( k \) 为次品数量,\( \lambda = n \times p_0 \),通过泊松分布的公式:

\[
P(X = k) = \frac{\lambda^k e^{-\lambda}}{k!}
\]

可以求出不同次品数量 \( k \) 下的概率。利用泊松分布的累积分布函数(CDF)计算 \( P(X \leq k) \),从而确定满足 95\% 和 90\% 信度下的阈值 \( k \)。

\section{示例计算}

假设企业抽取了 \( n = 100 \) 个零配件,供应商声明次品率 \( p_0 = 0.1 \),即标称次品率为 10\%。

1. 泊松分布的参数 \( \lambda \) 计算为:
   
\[
\lambda = 100 \times 0.1 = 10
\]

2. 根据泊松分布的累积分布函数和生存函数确定:

    - **95\% 信度下的拒收阈值**:当检测到 \( X \geq 15 \) 个次品时,拒收该批次零配件。
    - **90\% 信度下的接收阈值**:当检测到 \( X \leq 30 \) 个次品时,接收该批次零配件。

\section{总结}

使用泊松分布能够简化次品率检测问题中的抽样方案计算,特别是在次品率较低的情况下。通过设定拒收和接收的条件,可以在不同的信度下做出决策,帮助企业提高决策效率。

\end{document}
