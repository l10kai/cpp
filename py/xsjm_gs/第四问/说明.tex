\documentclass{article}
\usepackage{amsmath}
\usepackage{amsfonts}
\usepackage{graphicx}
\usepackage{geometry}
\usepackage{ctex}
\geometry{a4paper,scale=0.8}

\title{基于泊松分布的次品率计算}
\author{企业生产过程中的次品率决策}
\date{}

\begin{document}

\maketitle

\section{问题描述}

我们需要根据泊松分布重新计算表 1 和表 2 中的次品率。已知零配件、半成品和成品的次品率为某一初始值,通过抽样检测来估计实际次品率。

\section{泊松分布模型}

泊松分布的概率质量函数为:

\[
    P(X = k) = \frac{\lambda^k e^{-\lambda}}{k!}
\]

其中,\( \lambda = n \times p \),\( n \) 为抽样样本量,\( p \) 为假设的次品率。我们通过泊松分布生成的次品数量 \( k \) 来计算新的次品率:

\[
    \text{新的次品率} = \frac{k}{n}
\]

\section{表 1 和表 2 的次品率计算}

对于表 1 和表 2 中的各个零配件、半成品和成品,次品率通过以下方式计算:

1. 设定初始次品率 \( p_0 \),样本量 \( n \)。
2. 根据泊松分布生成次品数量 \( k \)。
3. 计算新的次品率 \( p_{\text{new}} = \frac{k}{n} \)。

表 1 和表 2 中的各项次品率根据上述公式重新计算,结果如下。

\section{总结}

通过使用泊松分布,能够较为精确地估计生产过程中各环节的次品率。这为企业制定最优决策提供了依据。

\end{document}
