\documentclass{article}
\usepackage[utf8]{inputenc}  % 输入字符集为UTF-8
\usepackage{ctex}  % 支持中文
\usepackage{amsmath}  % 支持数学符号

\begin{document}

\title{作业20241011}
\author{}
\date{}
\maketitle

\section{模型定义}

在存货管理中,我们使用经济订货量(EOQ)模型来确定最优订货批量和订货周期。

\subsection{1.1 不允许缺货模型}

假设:
\begin{itemize}
    \item \( D \): 年需求量
    \item \( S \): 每次订货成本
    \item \( H \): 每单位存货的年持有成本
    \item \( C \): 每单位商品的采购成本
\end{itemize}

总费用公式为:
\[
TC = \frac{D}{Q} S + \frac{Q}{2} H + DC
\]
其中:
\begin{itemize}
    \item 第一项为订货成本
    \item 第二项为持有成本
    \item 第三项为商品采购成本
\end{itemize}

\subsection{1.2 允许缺货模型}

在允许缺货的情况下,假设缺货成本为 \( C_{short} \),总费用公式变为:
\[
TC = \frac{D}{Q} S + \frac{Q}{2} H + DC + C_{short}
\]

\section{确定最优订货批量 \( Q^* \)}

\subsection{不允许缺货模型}

对总费用函数 \( TC \) 关于 \( Q \) 求导并设为零:
\[
\frac{d(TC)}{dQ} = -\frac{D}{Q^2} S + \frac{H}{2} = 0
\]
得到:
\[
Q^* = \sqrt{\frac{2DS}{H}}
\]

\subsection{允许缺货模型}

同样的方法,对允许缺货模型的总费用 \( TC \) 求导:
\[
\frac{d(TC)}{dQ} = -\frac{D}{Q^2} S + \frac{H}{2} = 0
\]
得到的最优订货批量为:
\[
Q^* = \sqrt{\frac{2DS}{H}}
\]

\section{确定最优订货周期 \( T^* \)}

最优订货周期 \( T^* \) 与最优订货批量相关:
\[
T^* = \frac{Q^*}{D}
\]

\section{结果比较}

在不允许缺货模型和允许缺货模型中,经过推导得到相同的最优订货批量 \( Q^* \) 和最优订货周期 \( T^* \)。这证明了在两种模型中,尽管采购成本和缺货成本的存在,最优的订货策略并没有改变。

\section{结论}

无论是在不允许缺货的情况下还是在允许缺货的情况下,增加采购货物本身的费用后,得出的最优订货批量和订货周期均保持一致。这表明,即使在不同的约束条件下,经济订货量模型依然能提供有效的决策支持。

\end{document}
