\documentclass{article}
\usepackage{amsmath}
\usepackage[utf8]{inputenc}  % 输入字符集为UTF-8
\usepackage{ctex}  % 支持中文
\usepackage{amsmath}  % 支持数学符号

\begin{document}

\title{三层玻璃与单层玻璃保暖效果比较}
\author{林凯}
\date{}
\maketitle

\section{基本概念}

热传导是指热量通过材料的导热性从高温区域传递到低温区域。热阻可以用来衡量材料对热传导的抵抗能力,通常由材料的厚度和导热系数决定。

\section{模型假设}

\begin{itemize}
    \item 假设单层玻璃和三层玻璃的导热系数相同,且厚度相同。
    \item 三层玻璃中,每层玻璃之间有一定的空气层,空气的导热系数较低。
    \item 忽略其他热损失(如对流、辐射)。
\end{itemize}

\section{热阻计算}

\subsection{单层玻璃的热阻}

单层玻璃的热阻 \( R_1 \) 可表示为:
\[
R_1 = \frac{d}{k}
\]
其中,\( d \) 为单层玻璃的厚度,\( k \) 为玻璃的导热系数。

\subsection{三层玻璃的热阻}

三层玻璃的总热阻 \( R_3 \) 可以表示为:
\[
R_3 = 3 \cdot \frac{d}{k} + 2 \cdot \frac{d_a}{k_a}
\]
其中,\( d_a \) 为空气层的厚度,\( k_a \) 为空气的导热系数。

\section{总热阻比较}

通过比较 \( R_1 \) 和 \( R_3 \),可以得出:
\[
R_3 > R_1
\]
这意味着三层玻璃的保暖效果优于单层玻璃。

\section{结论}

三层玻璃由于其更高的热阻,能够更有效地减少室内热量流失,因此在保暖效果上优于单层玻璃。空气层的存在显著提升了整体保温性能,尤其在寒冷环境中。

\end{document}
